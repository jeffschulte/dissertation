My doctoral work at Oregon State University naturally divides along
two avenues of research.  I have contributed to three peer reviewed
publications that fall within the realm of classical Density
Functional Theory, which is the primary research area of my advisor,
Dr. David Roundy.  Along side of this I have also completed a
reasearch project in the realm of systems biology, in which I simulate
the dynamics of the Min system of proteins within \emph{Escherichia
  coli}.

Contrary to how this may appear, these two lines of research actually
have quite a deal in common with each other.  The basic concepts are
really not too different, and many of the skills used in completing
one also applied to the other.  Both lines of research simulate
complex systems within a three dimensional grid.  In the case of DFT,
we calculate thermodynamic densities as a function of space, while in
the case of the Min project, we calculate protein densities, also as a
function of space.  The mathematics in both address interactions
between the local densities and their neighbor densities, and both
simulate microscopic things that result in macroscopic effects.  The
process of conducting both types of research requires working out
basic conceptual models, than coding up interactions, then managing
large data sets as the systems are simulated, and then plotting the
data in various ways to reveal properties of the systems.  Also, in
both projects we do the bulk of our simulating in c++, the bulk of our
plotting in python, and the data management using the same linux
tools.

In essence, the two lines of research were not very much different
than eachother at all, and I felt just as at home working in one than
I did the other.  I therefore include them both in this dissertation.

I'll lead off with the systems biology project, since this is the
project that is in the most complete sense 'mine', and since of the
two, the figures from this project are the prettiest.%%  The idea to do
%% it in the first place was mine, I was heavily involved with every step
%% of the conceptual development, I worked out and wrote the vast
%% majority of the simulation code, I wrote the majority of the actual
%% paper's text, and I did the vast majority of the data management.
%% This is not to take away from Rene Zeto, who helped me with the
%% project a great deal.  Rene did a large amount of very good work with
%% the plotting, and some editing and expanding of the simulation code.
%% He did his work fast and enthusiastically.


