Fange paper 2006:

The one difference between this reaction model and that of Huang is
that the cytoplasmic will bind to the membrane on it's own or
nonlinearly with MinD that is already bound there, but will not bind
preferentially to MinD:MinE complexes there.  The authors state that
they eliminate this reaction from the mkodel because there is no
biochemical support for this reaction being present.  While stochastic
versions have been poublished before Fange, they are the ones that
show that the experimental behavoir of certain MinD mutants is only
reproduced by stochastic simulation and not by deterministic
simulation.

The authors state that the reaction rastes in the model are not known,
so they vary the parameters extensively and pick the ones that best
recreate experimental results for the different cell shapes (of the
mutants).  They varied these extensively, having 100-fold values in
each of the five dimensions.  They test against $ftsZ^{-}$ cells,
which have the same cylindrical radius but are longer, from $10\mu m$
to $15\mu m$, and $rodA^{-}$ cells, which have a spherical shape of
approximate radius $1.5\mu m$.  They vary the parameters
systematically and observe in what conditions the mean field model
exhibits oscillations.  The stochastic version is more accurate in the
round cell shapes, showing a cluster of protein that is somewhat
random in where it groups on the walls (although in the immediate
sense it shows a bias twoards collecting agains the wall directly
opposite to it's last collection upon the wall.)

Also, they developed the MesoRD software and they cite it in ref[44]

They use MesoRD a reaction diffusion simulator, to contuct the
simulations.

They do say some words about phase drift, talk about how cells with
less molecules exhibit more (yet relatively minor) amounts of temporal
phase drift.  Not sure if want to say anything about this though.

Fange doesn't say what parameters

Halatek and Frey 2012 perameters:
\begin{align}
  D_{D}  = 16\mu m^2/s,   D_{E}  = 10\mu m^2/s,
  D_{d}, D_{de}  = .013\mu m^2/s,
  ATD->ATP = 6s^{-1}
  k_{de}(20degreesC) = .4s_{-1}
  k_{D} = .1\mu m s_{-1}
  k_{dD} = .108\mu m^2 s_{-1}
  k_{dE} = .435\mu m^2 s_{-1}
\end{align}
ADP->ATP is called nucleotide exchange rate
