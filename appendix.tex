\label{chapter:appendix}
The expression for the asymmetric correlation function
$g_\sigma^A(\rr)$ (Equation~\ref{eq:g-A-exact}) involves the
functional derivative $\frac{\delta A_{HS}}{\delta
  \sigma(\mathbf{r})}$.  In this appendix we will explain how this
derivative is evaluated.  We begin by applying the chain rule in the
following way:
  \begin{align}
    \frac{\delta A_{HS}}{\delta \sigma(\mathbf{r})} &=
    \int \left(
    \sum_\alpha
    \frac{\delta A_{HS}}{\delta n_\alpha(\mathbf{r}')}
    \frac{\delta n_\alpha(\mathbf{r}')}{\delta \sigma(\mathbf{r})}
    \right) d\mathbf{r}'
  \end{align}
This expression requires us to evaluate $\frac{\delta A_{HS}}{\delta
  n_\alpha(\mathbf{r}')}$ and $\frac{\delta
  n_\alpha(\mathbf{r}')}{\delta \sigma(\mathbf{r})}$.  The former is
straightforward, given Equations~\ref{eq:Phi1}-\ref{eq:Phi3}, and we
will write no more about it.  The functional derivatives of the
fundamental measures, however, require a bit more subtlety, and we
will address them here.

We begin with the derivative of $n_3$, the filling fraction, which we
will discuss in somewhat more detail than the remainder, which are
similar in nature.  Because the diameter $\sigma(\rr)$ is the diameter
of a sphere \emph{at position~$\rr$}, we write the fundamental measure
$n_3(\rr')$ as
\begin{align}
  n_3(\rr') &= \int n(\rr'') \Theta\left(\frac{\sigma(\rr'')}{2}
 -\left|\rr' - \rr''\right|\right)
  d\rr''
\end{align}
where we note that $\sigma(\rr'')$ and $n(\rr'')$ are the diameter and
density, respectively, of spheres centered at position~$\rr''$.  Thus the
derivative with respect to the diameter of spheres at position
$\rr$ is
\begin{align}
  \frac{\delta n_3(\rr')}{\delta \sigma(\rr)} &= \frac 12 \int n
  (\rr'') \delta\left(\frac{\sigma(\rr'')}{2} -\left|\rr' - \rr''\right|\right)
\delta(\rr-\rr'') d\rr'' \\ &= n (\rr) \delta(\sigma(\rr)/2
  -\left|\rr' - \rr\right|)
\end{align}
This pattern will hold for each fundamental measure: because we are
seeking the change in free energy when spheres at point~$\rr$ are
expanded, the integral over density is eliminated.  To compute the
correlation funtion $g_\sigma^A$, we convolve this delta function with
the product of the density and a local derivative of $\Phi(\rr)$:
\begin{align}
  \frac{\delta A_{HS}}{\delta \sigma(\rr)} &= \int \frac{\partial \Phi(\rr')}{\partial
    n_3(\rr')}n(\rr') \delta(\sigma/2-|\rr'-\rr|)d\rr'
  + \cdots
\end{align}
As we shall see, there are only four convolution kernels, leading to
four additional convolutions beyond those required for FMT.

The functional derivative of $n_2$ introduces our second convolution
kernel, which is a derivative of the delta function.
\begin{align}
  %n_2(\rr') &= \int n(\rr'') \delta(\sigma(\rr'')/2 -\left|\rr' - \rr''\right|) d \rr''\\
  \frac{\delta n_2(\rr')}{\delta \sigma(\rr)} &= \frac 12 n(\rr) \delta'(\sigma(\rr)/2 -\left|\rr' - \rr\right|)
\end{align}
The derivatives of the remaining scalar densities $n_1$ and $n_0$ reduce to
sums of the terms above:
\begin{align}
  %n_1(\rr') &= \int \mathbf{dr''} \frac{n(\rr'')}{2\pi \sigma(\rr'')}
  %\delta(\sigma(\rr'')/2 -\left|\rr' - \rr''\right|) \\
  \frac{\delta n_1(\rr')}{\delta \sigma(\rr)}
  = \frac{n(\rr)}{4\pi
    \sigma(\rr)}\delta'(\sigma(\rr)/2 -\left|\rr' - \rr\right|) -
  \frac{n(\rr)}{2\pi
    \sigma(\rr)^2}\delta(\sigma(\rr)/2 -\left|\rr' - \rr\right|)
\end{align}
and
\begin{align}
  %n_0(\rr') &= \int \mathbf{dr''} \frac{n(\rr'')}{\pi \sigma(\rr'')^2}
  %\delta(\sigma(\rr'')/2 -\left|\rr' - \rr''\right|) \\
  \frac{\delta n_0(\rr')}{\delta \sigma(\rr)}
  = \frac{n(\rr)}{2\pi
    \sigma(\rr)^2}\delta'(\sigma(\rr)/2 -\left|\rr' - \rr\right|) -
  2 \frac{n(\rr)}{\pi
    \sigma(\rr)^3}\delta(\sigma(\rr)/2 -\left|\rr' - \rr\right|)
\end{align}

The vector-weighted densities $\mathbf{n}_{V1}$ and $\mathbf{n}_{V2}$
give terms analogous to those of $n_1$ and $n_2$:
\begin{align}
  %\mathbf{n}_{V2}(\rr') &= \int n(\rr') \delta(\sigma(\rr'')/2 -\left|\rr' - \rr''\right|)
  %  \frac{\rr'-\rr''}{|\rr'-\rr''|} d \rr''\\
  \frac{\delta \mathbf{n}_{V2}(\rr')}{\delta \sigma(\rr)} = -\frac 12 n(\rr) \delta'(\sigma(\rr)/2 -\left|\rr' - \rr\right|)
    \frac{\rr-\rr'}{|\rr-\rr'|}
\end{align}
\begin{multline}
  %\mathbf{n}_{V1}(\rr') &= \int d\rr'' \frac{n(\rr'')}{2\pi \sigma(\rr'')}
  %\delta(\sigma(\rr'')/2 -\left|\rr' - \rr''\right|) \frac{\rr'-\rr''}{|\rr'-\rr''|}\\
  \frac{\delta \mathbf{n}_{V1}(\rr')}{\delta \sigma(\rr)}
  = -\frac{n(\rr)}{4\pi
    \sigma(\rr)}\delta'(\sigma(\rr)/2 -\left|\rr' - \rr\right|) \frac{\rr-\rr'}{|\rr-\rr'|}
  \\ +
  \frac{n(\rr)}{2\pi
    \sigma(\rr)^2}\delta(\sigma(\rr)/2 -\left|\rr' - \rr\right|) \frac{\rr-\rr'}{|\rr-\rr'|}
\end{multline}
Thus there are four convolution kernels used in computing $g_\sigma^A$:
one scalar and one vector delta function, and one scalar and one
vector derivative of the delta function.
