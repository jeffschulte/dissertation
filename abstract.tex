

This thesis reports on computational research in two different areas.
I first discuss the Min-protein system found within \emph{Escherichia
coli}.  Following this I discuss an extended investigation into
improving free energy functionals that are used within Classical
Density Functional Theory in order to model water.
%

Chapter~\ref{chapter:protein} examines the dynamics of the Min-protein
system within E. Coli, which aid in regulating the process of cell
division by identifying the center of the cell.  While this system
usually exhibits robust bipolar oscillations in a variety of cell
shapes, recent experiments have shown that when the cells are
mechanically deformed into wide, flattened out, irregular shapes, the
spatial regularity of these oscillations breaks down. We employ widely
used stochastic and deterministic models of the Min system to simulate
cells with flattened shapes.  The deterministic model predicts strong
bipolar oscillations, in contradiction with the experimentally
observed behavior, while the stochastic model, which is based on the
same reaction-diffusion equations, does predict spatially irregular
oscillations as observed in experiment.  We further report simulations
of symmetric but flattened cell shapes, and find that it is the
flattening rather than the asymmetry of the cell shapes that causes
the irregular oscillation behavior.
%

Chapter~\ref{chapter:intro} begins our discussion of our Classical
Density Functional Theory research by introducing many of the key
concepts used in the following chapters.
%

Chapter~\ref{chapter:contact} investigates the value of the
distribution function of an inhomogeneous hard-sphere fluid at
contact. This quantity plays a critical role in statistical
associating fluid theory, which is the basis of a number of recently
developed classical density functionals, including ones developed
within my research group. We define two averaged values for the
distribution function at contact and derive formulas for each of them
from the White Bear version of the fundamental measure theory
functional, using an assumption of thermodynamic consistency. We test
these formulas, as well as two existing formulas, against Monte Carlo
simulations and find excellent agreement between the Monte Carlo data
and one of our averaged distribution functions.
%

Chapter~\ref{chapter:saft} details our modifications our recently
published statistical associating fluid theory-based classical density
functional theory for water, incorporating this improved distribution
function at contact.  We examine the effect of this alteration by
studying two hard-sphere solutes and a Leonard jones approximation of
a krypton-atom solute, and find improvement.
%% We then modify our recently published statistical associating fluid
%% theory-based classical density functional theory for water,
%% incorporating this improved distribution function at contact.  Our
%% function replaces another in the association term of our free energy,
%% improving its description of hydrogen bonding. We examine the effect
%% of this improvement by studying two hard solutes (a hard hydrophobic
%% rod and a hard sphere) and a Lennard-Jones approximation of a krypton
%% atom solute. The improved functional leads to a moderate change in the
%% density profile and a large decrease in the number of hydrogen bonds
%% broken in the vicinity of the hard solutes. We find an improvement of
%% the partial radial distribution for a krypton atom in water when
%% compared with experiment.
%

Finally, Chapter~\ref{chapter:pair} introduces an approximation for
the pair distribution function of the inhomogeneous hard sphere
fluid. Our approximation makes use of the new distribution function at
contact refered to above. This approach achieves greater computational
efficiency than previous approaches by enabling the use of exclusively
fixed-kernel convolutions, which allows for an implementation using
fast Fourier transforms. We compare results for our pair distribution
approximation with two previously published works and Monte Carlo
simulation, showing favorable results.



