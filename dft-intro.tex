
Talk about why need functionals to deal with inhomogenous situations.
Then talk about how functional deriviatives are taken.

\section{Classical Density Functional Theory}

A classical statistical ensemble is a collection of an arbitrary
number of classical systems of particles that have the same
macroscopic properties and internal restrictions, but have positions
and momenta that are otherwise distributed randomly.  In the canonical
ensemble, the number of particles $N$, the total volume of the system
$V$, and the tempurature $T$ are constant.  In this ensemble the free
energy, defined as $F = U - TS$, where $U$ is the internal energy and
$S$ is the entropy of the system, is at a minimum in thermodynamic
equilibrium. In the grand canonical ensemble, the number of particles
is allowed to vary while $V$, $T$ and the chemical potential $\mu$ are
constant.  In this ensemble it is the grand potential, defined as
$\Omega = F - \mu N$, that is at a minimum in thermodynamic
equilibrium.

In the case of inhomogenous fluids, a treatment of the inhomogenous
external potential $\phi(\rr)$ restricts the positions of the
particles in the system and stands in the place of the volume $V$ in
the thermodynamic equations.  In such a system, a change in the
internal energy is
\begin{align}
  \delta U = T\delta S + \int \rho^{(1)}(\rr)\delta \phi(\rr)d(\rr) + \mu \delta N
\end{align}
where $\rho^{(1)}$ is the 'one particle density'.
$\rho^{(1)}(\rr)d\rr$ is the average number of particles at $\rr$
within a volume $d\rr$.  One can calculate $\rho^{(1)}(\rr)$ by taking
the one-particle limit of the n-particle density of the grand
canonical ensemble,
\begin{align}
  \label{eq:n-particle-density}
  \rho_N^{(n)}(\rr^n)=\frac{1}{\Xi}\sum^{\infty}_{N=n}\frac{exp(N\beta \mu)}{\Lambda^3 (N-n)!} \
  \int exp(-\beta V_N)d\rr^{(N-n)}.
\end{align}
$\rho_N^{(n)}(\rr^n)$ has the general form of a statistical mechanical
probability, with an integration over possible states of Boltzmann
factors divided by the grand canonical partition function,
\begin{align}
  \Xi = \sum_{N=0}^{\infty}\frac{exp(N\beta\mu)}{h^{3N}N!}\int\int exp(\beta \mathcal{H})d\rr^Nd\mathbf{p}^N.
\end{align}
$V_N$ in Eq.~\ref{eq:n-particle-density} is the total interaction
potential between all of the particles in the system, and the spatial
integral is taken over all the potential positions of the particles
\emph{other} than the $n$ particles at positions $\rr^n$.  The
$exp(N\beta \mu)$ term accounts for the chemical potential's
regulation of the average number of particles in the system.  When
implementing density functional theory, one adjusts $\mu$ in order to
restrict the system's average number of particles,
\begin{align}
  \int_{system} \rho_N^{(1)}(\rr)d\rr = \langle N \rangle
\end{align}
to be a desired value.

The probability density, which is closely related to the n-particles
density, is $f_0(\rr^N,\mathbf{p}^N;N)$.
$f_0(\rr^N,\mathbf{p}^N;N)d\rr^N$ is the probability that there are
$N$ particles in the system and that those particles are found within
the infinitesimal range of positions $\rr^N$ and momenta
$\mathbf{p}^N$.  It's definition is
\begin{align}
  \label{eq:prob-density}
  \mathit{f}_0(\rr^N,\mathbf{p}^N;N) = \frac{exp(-\beta(\mathcal{H}-N\mu))}{\Xi}
\end{align}
where $\Xi$ is once again the partition
function.

Classical Density Functional Theory assumes that the Hamiltonian can
be split into linearly combined parts:
\begin{align}\label{eq:hamiltonian}
  \mathcal{H}(\rr^N,\mathbf{p}^N) = K_N(\mathbf{p}^N) + V_N(\rr^N) + \Phi(\rr^N)
\end{align}
The three terms on the right are the kinetic energy, potential
interaction between particles, and external potential, respectively.
The kinetic energy is a function of only the momenta of the particles,
and the two potentials of only their positions.

Substituing this for the Hamiltonian term in Eq.~\ref{eq:prob-density}
and taking the natural logarithm, we have:
\begin{align}
  \label{eq:log-of-f}
  \ln \mathit{f}_0 = \beta\Omega - \beta K_N - \beta V_N - \beta \Phi_N + N\beta \mu
\end{align}
where we have used the relation
\begin{align}
  \Omega = -k_BT\ln\Xi,
\end{align}
which is the basic connection between thermodynamics and statistical
mechanics in the grand canonical ensemble.

Because the external potential $\Phi(\rr)$ and $\mu$ are both constant
and $\rho^{(1)}(\rr)$ is the average ensemble equilibrium density at
$\rr$, the two right most terms are
\begin{align}
  \langle \Phi_N \rangle = \int \rho^{(1)}(\rr)\phi(\rr)d\rr \
  and \
  \langle N\mu \rangle = \mu \int \rho^{(1)}(\rr) d\rr
\end{align}

Using these relations and switching terms around in
Eq.~\ref{eq:log-of-f} results in the equation
\begin{align} \label{eq:intrinsic-free-two}
  \langle K_N + V_N + k_BT \ln \mathit{f}_0 \rangle \
  = \Omega - \int \rho^{(1)}(\rr)\phi(\rr)d\rr + \mu \int \rho^{(1)}(\rr)d\rr
\end{align}

The thermodynamic definition of the grand potential gives us $F =
\Omega + \mu N$.  From this we can see that the right hand side of
Eq.~\ref{eq:intrinsic-free-two} is analougous to the free energy of
the inhomogenous system minus the energy due to the external
potential.  We call this function the 'intrinsic free energy' $\iF$
since it only includes the interaction energy between the particles
within the system:

\begin{align}
  \label{eq:intrinsic-free-three}
  \iF = \Omega - \int \rho^{(1)}(\rr)\phi(\rr)d\rr + \mu \int \rho^{(1)}(\rr)d\rr \
  = \langle K_N + V_N + k_BT \ln \mathit{f}_0 \rangle
\end{align}

It can be shown that for a given $\mu$, $T$, and defined function
$V_N$ describing the potential interaction between particles, there is
a one-to-one relation between the external potential function and the
equilibrium density profile $\rho^{(1)}$ at thermodynamic equilibrium.
The grand canonical density function $f_0$ (Eq.~\ref{eq:prob-density})
for a given function $V_N$, $\mu$, and $T$ is a function of only the
external potential, so as a result it is completely determined by
$\rho^{(1)}$.  $V_N$ is also wholly determined by $\rho^{(1)}$ and
$K_N$ is determined by tempurature.  Thus the right hand side of
Eq.~\ref{eq:intrinsic-free-three} is a function of only the external
potential and therefore also of only $\rho^{(1)}$, which means that
for a given $\mu$, $T$, and defined function $V_N$, $\iF$ is wholly
determined by $\rho^{(1)}$.

Thanks to all of this, we can write
\begin{align}
  \Omega_{\phi}[n] = \iF[n] + \int n(\rr)\phi(\rr)d\rr - \mu\int n(\rr)d\rr
\end{align}
where $n(\rr)$ is a density profile that we will adjust systematically
in order to minimize $\Omega_{\phi}[n]$.  When $\Omega_{\phi}[n]$ is
minimized, it becomes the minimized grand potential of the system, and
the density profile $n(\rr)$ of the system is the density profile in
thermodynamic equilibrium, $\rho^{(1)}$.  It is finding this density
profile in thermodynamic equilibrium, and then the properties that can
be derived from it, that is the ultimate goal of Classical Density
Functional Theory.  Throughout the rest of this dissertation I will be
using $n(\rr)$ to refer to the particle density of the system, and the
reader should understand that it is a function that is meant to be
adjusted in order to minimize the grand potential energy function.

We begin the process of Classical Density Functional Theory by
designing the functional form of the intrinisic free energy $\iF$.
This is the major theoretical part of the process.  We then decide
upon an external potential $\phi(\rr)$ that defines the external
restrictions of the system, set the tempurature $T$ and chemical
potential $\mu$, and systematically adjust the density profile until
we have found the global minimum of this grand potential.  I won't
discuss the process of finding the density profile that minimizes the
grand potential in this dissertation, primarily because the algorithms
that do this were all in place before I joined the lab!  My work is
centered around the first part of the process, namely constructing the
functional form of $\iF$.



\clearpage
\newpage



\section{Introduction to SAFT and explanation of first free energy term}

\fixme{do more of an introduction to SAFT, taken from papers}

Work on Classical Density Functional Theory for inhomogeneous fluids
involves creating the intrinsic free energy functional of the density
profile, $\iF[n]$.  This is comprised of a sum of terms that
individually address different conceptual aspects of the system.
Terms that treat different types of interaction between particles are
added to a term that describes the reference system.  The functional
that we use in our work is called the Statistical Associating Fluid
Theory or SAFT free energy, and it is broken into the following terms:

\begin{align}
  \iF_{SAFT} = F_{ideal} + F_{hard sphere} + F_{dispersion} + F_{association}
\end{align}
We will discuss the meaning of these terms below in much more detail,
but write it down now to illustrate the general structure of the total
functional.


Although we work specifically with the very popular SAFT functionals
in current development, our contributions described in the next
chapters are applicable to many different types of fluid functionals.
The next chapter details a function that we've created, the
distribution function at contact $g_\sigma(\rr)$, which treats
correlations between particles within a system, while the following
chapter discusses it's use within a specific SAFT functional and it's
effect upon that functional's results.  The chapter after that
introduces another function, the pair distribution function
$g^{(2)}_{HS}(\rr_1,\rr_2)$, that is closely related to the first.
It's important for the reader to recognize that while we discuss these
functions in terms of their place within the SAFT free energy, they
and their conceptual founding are useful to other Classical Density
Functional Theories.



The first two terms, $F_{ideal}$ and $F_{hard~sphere}$, are typically
thought of as the functional's reference system terms.  The
$F_{ideal}$ is ubiquitous, and treats particles that have no
interactions between them.  While the $F_{hard~sphere}$ term does
treat particle interactions, the type of interaction it treats is so
basic and simple that this term is also seen as a reference term and
is incorporated into many very different functionals.  I give a
detailed introduction to the particular functional that we use for the
$F_{hard~sphere}$ term below (that of the White Bear free energy),
because while we don't actually modify it in our implementation of
inhomogenous SAFT, we do draw heavily from its ideas when creating our
functions.  The rest of the terms in $\iF$ address different types of
interactions between particles.  SAFT itself departs from other
theories in these last two terms.  In inhomogenous fluids, each one of
these terms is itself a functional of the density profile.


The first term in the SAFT functional is the ideal free energy term
$F_{ideal}$, and it treats a system of particles that do not interact
with eachother.  This is an obvious place to start if one is to build
the description of particle interaction terms upon a reference system.
It's lack of interaction actually causes this term to be the only one
that we can construct exactly, with no approximations.  To see why, we
point out that a system of non-interacting particles is able to
satisfy what is usually called the the local density approximation
(although we shouldn't call it this here because for non-interacting
particles it's not an approximation!)  The idea is that the free
energy functional can be written as an integral of a completely local
function of the density profile:
\begin{align}
  \iF_{local~density}[n] = \int f(n(\rr)) d\rr
\end{align}
where $f(n(\rr))$ is the free energy per unit volume of a homogenoues
fluid at a density $n(\rr)$.  In essence, each bit of volume becomes
it's own thermodynamic system, with a free energy equal that of a
homogenous density of particles at the local density, and the free
energies from all the bits of volume are added up to get the total.
The construction neglects any interaction between the particles, so
that any spatial variation in the density will be due entirely to the
varying external potential.  As an approximation for interacting
fluids, it does in fact apply to external potentials that modulate the
density slowly over space, much more slowly than correlation lengths.
It has thus been used in the past to construct entire intrinsic free
energy functionals beyond the reference $F_{ideal}$ term (cite).  It
breaks breaks down rather quickly, however, near hard walls for
interacting systems, where often the spheres will stack up in 'layers'
(should see some figure) and the local densities become greater than
bulk freezing densities (cite).

However, $F_{ideal}$ can be constructed exactly in this integral of
local density fashion, so all we need to do is integrate the free
energy of an ideal homogenous system at the local density.  Going back
to basic thermodynamics and statistical mechanics, we have:
\begin{align}
  F = -k_BT \ln Q_N = -k_BT \ln (\frac{V}{N!\Lambda^3})
\end{align}
where $Q_N$ is the partition function and $\Lambda$ is de Broglie
thermal wavelength.  Using th Stirling approximation for $N!$, we have
\begin{align}
F^{id} = Nk_BT(\ln \Lambda^3 \rho - 1))
\end{align}
Taking this as the free energy per unit volume and integrating, we
have
\begin{align}
  F_{ideal}[n] = \frac{1}{\beta}\int d\rr n(\rr)(\ln (\Lambda^3 n(\rr))-1)
\end{align}
This will be our reference $F_{ideal}$ term, as is common within
Clasical Density Functional Theory.

\clearpage
\newpage

\section{Virial Equation, Mayer functions, and the Carnahan Starling Equation}

The second term in the SAFT intrinsic free energy functional,
$f_{hard~sphere}$, is generally refered to as a reference term and the
specific functional that we use for this term is also used in many
other density functional theories.  I'll introduce the functional and
the theory behind it in some detail in the next section because it is
so widely used and, more importantly, because an understanding of the
ideas involved is neccesary for an understanding of our own work.
Before describing the term itself, however, I'll explain in this
section some important things that lead up to this theory, namely the
Virial equation, Mayer functions, and the Carnahan Starling Equation.

The Virial Equation applies to homogeneous fluids.  It equates a
thermodynamic, intensive quantity (likes the pressure) to an expansion
of the homogeneous density of the fluid.  It's standard form is
\begin{align}
  \label{eq:virial-expansion}
  \frac{\beta P}{\rho} = 1 + \sum_{i=1}^{\infty}\beta_i\eta^i.
\end{align}
where
\begin{align}
  \eta = \frac{\pi \rho \sigma^3}{6}
\end{align}
is the 'packing fraction'.  $\sigma$ here is the diameter of the
spherical particles of the fluid.  The packing fraction is really just
a more convenient way to refer to the density.  We use it extensively
throughout our work.

The expansion of Eq.~\ref{eq:virial-expansion} comes out of a
formulation of the partition function that is most often expressed as
a series of diagrams that have well defined rules of construction.  I
won't explain the diagrams or their rules here, but Figure(REF) shows
an example so that if the reader sees them somewhere she'll know what
they are.  A single term (or diagram) in this expansion of the
partition function is in fact a spatial integral of particle densities
multiplied by a number of what are called Mayer Functions.  I'll give
a loose outline of an explanation here, since a basic idea of where
this expansion comes from should be useful to the reader.  The
derivation starts with the partition function:
\begin{align}
  \Xi &= \sum_{N=0}^{\infty}\frac{exp(N\beta\mu)}{h^{3N}N!}\int...\int exp(-\beta \mathcal{H})d\rr^Nd\mathbf{p}^N \\
  &=\sum_{N=0}^{\infty}\frac{exp(N\beta\mu)}{h^{3N}N!}\int...\int exp(-\beta (V_N + \phi(\rr))d\rr^Nd\mathbf{p}^N
\end{align}
where $V_N$ is the interaction potential between all the particles in
the system and $\phi(\rr)$ is the external potential.  If the the
interaction potential can be written as a summation of pairwise
superposable interactions, i.e.
\begin{align}
  V_N = \sum_{i<j}^{all~particles} v(i,j)
\end{align}
 than the partition function can be written as
\begin{align}
  \Xi = \sum_{N=0}^{\infty}\frac{1}{N!}\int... \int \left(\overset{N}{\underset{i<j}{\Pi}} f(i,j)\right)
  \left( \overset{N}{\underset{i=1}{\Pi}} \frac{exp(\beta (\mu - \phi(\rr)))}{\Lambda^3}\right)d\rr^N
\end{align}
where $f(i,j)=exp(-\beta v(i,j))$ is the Mayer function between two
particles.  With the help of the diagrams, one can take the natural
logarithm of the partition function to get the grand potential, and
then derivatives to find what are called direct correlation functions
(which I won't explain here). The use of an identity yields a
relationship between the chemical potential and the density,
\begin{align}
  \beta \mu = \beta \mu^{id} - \sum_{i=1}^{\infty}\beta_i\rho^i.
\end{align}
Then the use of the relation from thermodynamics,
\begin{align}
  \left(\frac{\partial P}{\partial \rho}\right)_T \
  = \rho \left(\frac{\partial \mu}{\partial \rho}\right)_T,
\end{align}
allows one to express the pressure as
\begin{align}
  \frac{\beta P}{\rho} = 1 + \sum_{i=1}^{\infty}\beta_i\eta^i
\end{align}
where
\begin{align}
  \label{eq:virial-coeff}
  \beta_i = (\frac{6}{\pi \sigma^3})^i B_{i+1}.
\end{align}

The virial formulation of thermodynamic properties is useful, but it
requires an expansion of coeffients, which can be a nuisance
computationally.  Carnahan and Starling (cite) were able to develop a
rule for coefficient generation that approximates
Eq.~\ref{eq:virial-coeff}, but that results in integers that one can
allow for a geometric the one written above, based on and then
multiple terms of a The virial coefficients $\beta_i$ based on the
procedure above are not integers, so in order to get get to a
thermodynamic property the series must be expanded.  Carnahan Starllng
constructed a method that allows them to approximate each coefficient
by a very close integer in a sysematic fashion that allows them to
create a geometric series.  This series yields a simple, analytic
approximation of the pressure in the homogenous fluid as a function of
density 3.9.17


\begin{align}
  \frac{\beta P}{\rho}=\frac{1+\eta+\eta^2-\eta^3}{(1-\eta)^3}
\end{align}
or for the free energy:
\begin{align}
  \frac{\beta F^{ex}}{N}=\frac{\eta(4-3\eta)}{(1-\eta)^2}
\end{align}

Its is very successful in predicting pressure of homogenous hard
sphere fluid at different densities.

White Bear uses MCSL equation of state, which is the generalization to
the multi component mixture of hard spheres of the Carnahan-Starling
equation of state.  We describe the Carnahan-Starling equation of
state here because we use it in the first chapter.





\section{$F_{hard~sphere}$, Fundemental Meaure Theory, and White Bear}

We now return to analyzing the terms within SAFT:
\begin{align}
  F_{SAFT} = F_{ideal} + F_{hard~sphere} + F_{other terms}
\end{align}

Remember that $F_{ideal}$, because it treats particles that do not
interact with one another, can be described as an intagral of a purely
local free energy function over the volume of the system.  In fact,
this one term addresses any aspect of the system that is
non-interactive, in the sense that every other term in $F_{SAFT}$ is
designed to specifically deal with a different type of potential
interaction among the particles.  Thus, $F_{ideal}$ is in a sense the
most basic reference term in the system.  However, as I've explained
above, the seconds term, $F_{hard~sphere}$, which we will describe here,
is also considered to be a reference system.

The potential interaction it describes is based on the fact that every
particle has at it's core a 'hard-core' repulsion to any other
particle.  In other words, as two particles approach each other in
space, there is a sudden, sharp spike in potential that prevents the
two particles from 'overlapping'.  The forces here can be complicated,
deriving from nuclear forces and the exclusion principle, but our
classical theories seek to approximate these in simple ways.

The hard sphere potential interaction is characterised by an
impenetrable spherical volume that is centered at a particles'
position.  The potential between two of these hard spheres
discontinuously jumps from zero to infinity when the spheres are a
distance apart that is equal to their combined radii:
\begin{align}
  v(r) &= \infty,~~ r < r_A + r_B \\
  &= 0,~~ r > r_A + r_B
\end{align}
where $r$ is the distance between the two particles, and $r_A$ and
$r_B$ are the radii of the two particles.  This hard sphere potential
that we use is not the only commonly used method for treating the
hard-core repulsion between particles.  The Leonard-Jones potential,
for example, another widely used potential energy description,
approximates the repulsion with a positive term porportional to
$\frac{1}{r^{12}}$



Once we decide upon a form for the potential, we must turn to the much
more difficult step of designing a free energy functional that is
appropriate for hard sphere particles.  One of the first methods by
which people attempted to deal with these interactions (not limited to
hard spheres) was to modify the form of the local density
approximation free energy discussed above,
\begin{align}
  \iF_{local~density}[\rho^{(1)}] = \int f(\rho^{(1)}) d\rr
\end{align}
to incorporate information about the particle densities immediately
surrounding each point.  They did this by redefining the density at
each point to be a convolution of the surrounding density with a
weighting function:
\begin{align}
  \label{eq:convolution-of-density}
  \bar{\rho}(\rr) = \int w(|\rr-\rr'|) \rh(\rr') d\rr'
\end{align}
and then integrating this density over space to get the total free
energy, in the same fashion (NOT THE SAME LOOK AT EQUATION) as the
local density approximation:
\begin{align}
  \iF[\rh] = \int \frac{f^{ex}(\bar{\rho})}{\bar{\rho}} \rh(\rr') d\rr'
\end{align}
Eq.~\ref{eq:convolution-of-density} allows one to shape the
approximately interaction, in a sense, by changing the structure of
the weighting function.  For example, if one were to choose for the
weighting function the step function $\Theta(|\rr-\rr'|)$, than the
modified density would incorporate in an equal way all the density
within a sphere surrounding the particle.  This is an extremely simple
example of a weighted density approximation.  Practical theories
formulated in this manner become very complicated.  We introduce it
here so that the reader knows what it is and also because the central
idea described by Eq.~\ref{eq:convolution-of-density} is used in FMT
as well.

Fundamental Measure Theory, created by Rosenfeld in 1989, also defines
the free energy in terms of a series of convolutions of densities, but
it is a considerable departure from the weighted density approximation
theories.  It is based on an involved derivation worked out by Perkis
and Yevik in (YEAR) of an equation of state for a homogeneous fluid.
Like the derivation of the Virial Expansion and Carnahan-Starling
Equation discussed above, this derivation also dealt with correlation
functions and ultimately expressed thermodynamic properties in terms
of homogeneous density.  Comparing this theory with the derivations of
another theory call Scaled Particle Theory, which once again I will
not explain but will say that it has to do conceptually with a system
of spheres and a growing cavity in which they are not allowed to go,
Rosenfeld recognized that the density sides of the Perkis-Yevick
equations can be reformulated in terms of convolutions of functions
that are constructed to describe the geometric properties of spheres.
He then discovered that for inhomogeneous systems, he could write down
the intrinsic free energy in terms of convolutions of densities with
these functions in such a way that the Percis-Yevik correlation
equations were reproduced in the limit of homogeneous density.  The
derivation is certainly an involved (and brilliant) one, and there is
actually another derivation to get to the same place introduced by
(TARAZONA? ) in (YEAR) that is thought by some to be more elegant.
However I don't use any ideas that are directly pulled from these
derivations in my work, besides those that can be taken from the
general characteristics (spherical nature) of the resulting Free
Energy functional.

The result of all this is an intrinsic free energy which neccasarily
has the form:
\begin{align}
  \label{eq:F-FMT-form}
  \iF_{hard~sphere}[\rho^{(1)}] = \int \Phi({n_i(\rr')})d\rr'
\end{align}
The integrand $\Phi$ here is a local function of $n_i(\rr)$, which are
the non-local convolutions of density with sphere-like weighting
functions discussed above.  In FMT they are referred to as
'fundamental measures'.  They are defined as:
\begin{align}
  n_3(\rr) &= \int n(\rr') \Theta(\sigma/2 -\left|\rr - \rr'\right|)
  d\rr' \label{eq:FMn3} \\
  n_2(\rr) &= \int n(\rr') \delta(\sigma/2 -\left|\rr - \rr'\right|) d\rr' \\
  \mathbf{n}_{2V}(\rr) &= \int n(\rr') \delta(\sigma/2 -\left|\rr - \rr'\right|) \frac{\rr-\rr'}{|\rr-\rr'|}d\rr'
\end{align}
\begin{align}
  \mathbf{n}_{V1} = \frac{\mathbf{n}_{V2}}{2\pi \sigma}, \quad
  n_1 &= \frac{n_2}{2\pi \sigma} , \quad
  n_0 = \frac{n_2}{\pi \sigma^2} \label{eq:FMrest}
\end{align}
We can see the spherical nature of the theory by inspecting the $n_i$.
The $n_3(\rr)$ weighting function is a step function that is designed
so that the density is integrated over the volume of a sphere of
radius $\sigma/2$, but will make no contributions outside of this
sphere.  $n_2$ only allows for integration of densities on the surface
of a sphere, but incorporates no density within or outside of it.
$n_{2V}$ is a vector version of $n_2$, and the others are different
versions of the same, modified to have different units.
  
The form of $\Phi({n_i(\rr)})$ is restricted to a certain extent by
dimensional analysis (the units of each term have to be right!)
throughout the derivation, but even given this the theory allows for a
certain amount of freedom in it's design.  Since his derivation of FMT
was originally based on concepts in the Perkis-Yevik derivation of the
equation of state, Rosenfeld constructed the form of his functional so
that in the limit of homogeneous density, the free energy of the
system approaches that given by the Perkis-Yevik equation of state.

While the theory has been a resounding success (MAYBE MENTION HOW
SUCCESSFUL IN BEGINNING OF SECTION), the use of the Perkis-Yevik
equation of state as the underlying, homogeneous limit equation causes
some problems.  For the hard sphere fluid this equation predicts a
pressure that is too high as bulk freezing tempuratures are
approached.  At the same time, the construction of FMT obeys a theorem
called the 'contact value theorem,' which states that the pressure at
a wall is equal to the temperature multiplied by the density in
contact with that wall, $p=kT\rho_{contact}$.  This theorem is very
important to our own work and I will discuss it in detail below, but I
mention it now to say that the Perkis-Yevik equation consequently
overestimates the density at contact at freezing bulk temperatures.
This is problematic, since much of the reason we use classical density
functional theory in analyzing inhomogeneous fluids in the first place
is to estimate what happens at walls!

In (YEAR) Roth et al addressed this issue in their version of FMT
which they named 'White Bear', anecdotally after a pub that they
frequented while writing the theory.  They keep the same general form
of Rosenfeld's FMT but adjust it so that in the limit of homogeneous
density it's free energy approaches that of the
Mansoori-Carnahan-Starling-Leland equation, a modified version of the
Carnahan-Starling equation of state, which we discussed above and
which we use directly in our own work.  The Carnahan-Starling is more
accurate in limiting cases than the Perkis-Yevik, so that the White
Bear hard-sphere free energy functional, $F_{hard~sphere}$, is overall
a more accurate one.  It is therefore the hard sphere free energy
reference system that we choose to use in our own work.

Below is the entire energy functional written in terms of the
fundamental measure, $n_i(\rr)$:
\begin{equation}
F_\textit{hard~sphere}[n] = k_B T \int \left(\Phi_1(\rr) + \Phi_2(\rr) + \Phi_3(\rr)\right) d\rr \; ,
\end{equation}
with integrands
\begin{align}
\Phi_1 &= -n_0 \ln\left( 1 - n_3\right) \label{eq:Phi1}\\
\Phi_2 &= \frac{n_1 n_2 - \mathbf{n}_{V1} \cdot\mathbf{n}_{V2}}{1-n_3} \\
\Phi_3 &= (n_2^3 - 3 n_2 \mathbf{n}_{V2} \cdot \mathbf{n}_{V2}) \frac{
  n_3 + (1-n_3)^2 \ln(1-n_3)
}{
  36\pi n_3^2\left( 1 - n_3 \right)^2
} , \label{eq:Phi3}
\end{align}


\section{The convolution theorem}

One of the largest advantages to using Fundamental Measure Theory, and
one that is not immediately obvious, is that the convolutions that
combine to construct the functional allow for very efficient
computation.  This is not very intuitive, since an integral of a
convolution integral must integrate over two dimensions, e.g.
\begin{align}
\int(f\ast g)(\xx)d\xx = \int \int f(\yy)g(\xx-\yy)d\yy d\xx
\end{align}
so that it may seem that the size of the computation would scale as
$N^2$, where $N$ is the size of the system.  It is true that in the
case of FMT, the weighting functions cut off the integrals at the size
on the order of a sphere of particle radius, but this can still be a
large enough volume so that a double integral for which one of the
volumes is this size and the other is the size of the whole system
would be too costly for practical computation.  FMT is saved, however,
by what is called the Convolution Theorem.  The Convolution Theorem
states that when one takes the Fourier Transform on a spatial
convolution of two functions (a double integral in space), the result
is two separate integrals over k-space that are simply multiplied
together.  (SHOW IN EQ FORM) When minimizing our functional, after
taking a Fourier Transform of the convolutions, we have merely to
integrate once over the one variable $\kk$.  Wait!  You may say.  This
is a cheat, since the Fourier Transform is itself an integral, so at
the end of the day we're still performing two integrals.  This would
be true if it were not for the computational technique known as Fast
Fourier Transforms, developed by (cite).  I won't explain how it works
here, but it's effect is to perform the transform, which would
normally have a computational cost on the order of N, with a
computational cost on the order of $\ln N$ instead.  Thus, when we
Fourier Transform the convolutions in the functional and proceed to
take the single integral over the result (in k-space), the
computational cost scales as $N \ln N$.  For large systems (we
simulate systems with hundreds or even thousands of particles) this
can certainly be the difference between practically possible and
impossible computations.  The proof is short and pretty so I'll relate
it here:
\begin{align}
\hat{h}(\kk) = \hat{f}(\kk)\hat{g}(\kk)
\end{align}
Is the convolution theorem.  Proof:
\begin{align}
h(\xx) &= (f\ast g)(x) = \int f(\yy)g(\xx-\yy)d\yy \\
\hat{f}(\kk) &= \int f(\yy) exp(-i2\pi \kk \cdot \yy)d\yy \\
\hat{g}(\kk) &= \int g(\yy) exp(-i2\pi \kk \cdot \yy)d\yy \\
\int h(\rr) exp(-i2\pi \kk \cdot \zz) d\zz &= \int \int f(\yy) g(\zz-\yy) exp(-i2\pi \kk \cdot \zz) d\yy d\zz
\end{align}
because the two integrals are necessarily over all space,
\begin{align}
\xx &= \zz - \yy \\
\hat{h}(\kk) &= \int \int f(\yy) exp(-i2\pi \kk \cdot \yy ) d\yy g(\xx) exp(-i2\pi \kk \cdot \xx) d\xx \\
&= \hat{f}(\kk) \hat{g}(\kk)
\end{align}


\section{Contact Value Theorem}

Fixed number of particles:
\begin{align}
F &= U - TS \\
dU &= TdS -pdV \\
dF &= dU - TdS - SdT = -SdT - pdV
\end{align}
picture a fixed volume except a little protrusion into it, so call
$F_{out}$ and $F_{flat}$.  So looking at partition functions:
\begin{align}
Z_{flat} = \int_V... \int_V exp(-\phi \{\rr^N \})d\rr^N
Z_{out} =  \int_{V-dV}... \int_{V-dV} exp(-\phi \{\rr^N \})d\rr^N
\end{align}
Writing $Z_{flat}$ in terms of $Z_{out}$:
\begin{align}
Z_{flat} = &\int_{V-dV}... \int_{V-dV} exp(-\phi \{\rr^N \})d\rr^N \notag \\
+ &\int_V... \int_V \int_{V-dV} exp(-\phi \{ \rr^N \})d\rr^N \
+ \int_V ... \int_{V-dV} \int_V exp(-\phi \{ \rr^N \})d\rr^N ... \notag \\
+ &\int_V... \int_{V}\int_{V-dV} \int_{V-dV} exp(-\phi \{\rr^N \})d\rr^N \
+ \int_V... \int_{V-dV}\int_V \int_{V-dV} exp(-\phi \{\rr^N \})d\rr^N ...
\end{align}
The term on the first line is just $Z_{out}$.  The terms on the second
line treat one particle being in the bit of volume and the others
being integrated over the rest of the volume.  Becuase every particle
is identical, it doesn't matter which is in the little bit of volume,
each of these terms is the same.  So they can be replaced by one of
them multiplied by $N$.  The terms on the last line or any line after
are of order 2 or higher in $dV$.  We will throw away these terms, and
there are two arguments that allow us to do this.  The simplest one is
that $dV$ is small, and so terms with a two $dV$s multiplied together
will be small.  One must be careful with this, however, because one is
really intagrating over the boltzmann factor over these volumes.  If
the interaction potential between the particles is constructed so that
they are highly attracted to eachother, than a state in which there
are two particles in the bit of volume can have the same order of
magnitude probability as the one in which there is just one.  In this
case we would not be able to say for certain that these terms are so
much smaller than the order 1 $dV$ terms on the seconds line that we
could reasonably ignore them. Thus the validity of the derivation has
to do with the size of your smallest measurement along the wall, and
the nature of the attractive potentialbetween the particles.  In our
case we deal with an interaction between hard spheres, which simply
exclude other spheres from being too close to them.  Thus it's
reasonable for us to imagine that if there is one hard sphere in the
bit of volume, than there is only the one, and any terms that address
the situation in which there are two in the bit of volume can be
ignored.  After applying these areguments we have:
\begin{align}
  \label{eq:zflat}
  Z_{flat} = Z_{out} + N \int_V... \int_V \int_{V-dV} exp(-\phi \{ \rr^N \})d\rr^N
\end{align}

Now we consider the statistical mechanical definition of the particle
density,
\begin{align}
  n(\rr) = \frac{N \int_V... \int_V exp(-\phi \{ \rr^N \})d\rr^{N-1}}{\int_V... \int_V exp(-\phi \{ \rr^N \})d\rr^N}
\end{align}
Which is the sum of the boltzmann factors of all the states for which
a particle is at position $\rr$, divided by the partition function for
the system.  Comparing this with the right-most term in the
Eq.~\ref{eq:zflat} we see that:
\begin{align}
 N \int_V... \int_V \int_{V-dV} exp(-\phi \{ \rr^N \})d\rr^N  &= Z_{flat} \int_{dV} n(\rr) d\rr
 &\approx Z_{flat} n(\rr)dV
\end{align}
where on the right we make the approximation that because the volume
is small $n(\rr)$ is constant over it.
Thus we have
\begin{align}
Z_{flat} = Z_{out} + Z_{flat} n(\rr)dV \\
Z_{out} = Z_{flat} (1-n(\rr)dV)
\end{align}

We have
\begin{align}
  dF = F_{out} - F_{flat} &= -kT \ln \left( \frac{Z_{out}}{Z_{flat}} \right) \notag \\
  &= -kT \ln \left( \frac{Z_{flat} (1-n(\rr)dV)}{Z_{flat}} \right) \notag \\
  &= -kT \ln \left( 1-n(\rr)dV \right) \notag \\
  &= kTn(\rr)dV
\end{align}
Then relating back to thermodynamics:
\begin{align}
  pdV &= dF = ktn(\rr)dV \\
  \label{eq:contact-value-theorem}
  p &= ktn(\rr)
\end{align}
Eq.~\ref{eq:contact-value-theorem} is the standard formulation of the
contact value theorem.


Another equation for the particle density is:
\begin{align}
n(\rr) = \frac{\delta \iF}{\delta V_{ext}}
\end{align}

\clearpage
\newpage




\section{Remaining terms in SAFT}
