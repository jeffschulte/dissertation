Jeff Going to reverse engineer the explanation of DFT:

Talk about why need functionals to deal with inhomogenous situations.
Then talk about how functional deriviatives are taken.

\section{Classical Density Functional Theory}

A statistical ensemble is a collection of an arbitrary number of
systems that have the same macroscopic properties and internal
restrictions, but have positions and momenta that are otherwise
distributed randomly.  In the canonical ensemble, the number of
particles $N$, the total volume of the system $V$, and the tempurature
$T$ are taken as constant.  In this ensemble the free energy, defined
as $F = U - TS$, where $U$ is the internal energy and $S$ is the
entropy of the system, is  minimized in thermodynamic equilibrium. In
the grand canonical ensemble, the number of particles is allowed to
vary while $V$, $T$ and the chemical potential $\mu$ are held constant.
In this ensemble it is the grand potential, defined as $\Omega = F -
\mu N$, that is minimized in thermodynamic equilibrium.

In the case of inhomogenous fluids, an inhomogenous external potential
$\phi(\rr)$ restricts the particles in the system and takes the place
of the volume $V$ in the thermodynamic equations.  In such a system, a
change in the internal energy is given by
\begin{align}
  \delta U = T\delta S + \int \rho^{(1)}\delta \phi(\rr)d(\rr) + \mu \delta N
\end{align}

In the grand canonical ensemble, the probability density
$f_0(\rr^N,\mathbf{p}^N;N)$ is defined so that
$f_0(\rr^N,\mathbf{p}^N;N)d\rr^N$ is the probability that there are
$N$ particles in the system and that those particles are found within
the infinitesimal range of positions $\rr^N$ and momenta
$\mathbf{p}^N$.  It's definition is
\begin{align} \label{eq:prob-density}
  \mathit{f}_0(\rr^N,\mathbf{p}^N;N) = \frac{exp(-\beta(\mathcal{H}-N\mu))}{\Xi}
\end{align}
where
\begin{align}
  \Xi = \sum_{N=0}^{\infty}\frac{exp(N\beta\mu)}{h^{3N}N!}\int\int exp(\beta \mathcal{H})d\rr^Nd\mathbf{p}^N
\end{align}
is the partition function of the grand canonical
ensemble. Eq.~\ref{eq:prob-density} is structured as the boltzmann
constant with an energy argument comprised of the Hamiltonian and
chemical potential, divided by the grand partitian function.

Classical Density Functional Theory assumes that the hamiltonian can
be split into linearly combined parts:
\begin{align}\label{eq:hamiltonian}
  \mathcal{H}(\rr^N,\mathbf{p}^N) = K_N(\mathbf{p}^N) + V_N(\rr^N) + \Phi(\rr^N)
\end{align}
The three terms on the right are the kinetic energy, potential
interaction between particles, and external potential, respectively.
The kinetic energy depends soley on the momenta of the particles, and
the two potentials soley on their positions.

Substituing this into the Hamiltonian term in
Eq.~\ref{eq:prob-density} and taking the natural logarithm, we have:
\begin{align}
  \ln \mathit{f}_0 = \beta\Omega - \beta K_N - \beta V_N - \beta \Phi_N + N\beta \mu
\end{align}
where we have used the relation
\begin{align}
  \Omega = -k_BT\ln\Xi
\end{align}
which is the basic connection between thermodynamics and statistical
mechanics in the grand canonical ensemble.

Because the external potential and $\mu$ both don't change and
$\rho^{(1)}(\rr)$ is the average ensemble equilibrium density at
$\rr$, it's true that
\begin{align}
  \langle \Phi_N \rangle = \int \rho^{(1)}(\rr)\phi(\rr)d\rr \
  and \
  \langle N\mu \rangle = \mu \int \rho^{(1)}(\rr) d\rr
\end{align}

Using these relations and switching terms around in Eq.~{}results in the equation
\begin{align} \label{eq:intrinsic-free-two}
  \langle K_N + V_N + k_BT \ln \mathit{f}_0 \rangle \
  = \Omega - \int \rho^{(1)}(\rr)\phi(\rr)d\rr + \mu \int \rho^{(1)}(\rr)d\rr
\end{align}


Switching around the definition of the thermodynamic grand potential
gives us $F = \Omega + \mu N$.  We can see the right hand side of
Eq.~\ref{eq:intrinsic-free-two} is analougous to the free energy of the
inhomogenous system minus the part given by the external potetnial.
We call this function the 'intrinsic free energy' $\iF$ since it
addresses the interactions between the particles within the system:

\begin{align}
  \iF = \Omega - \int \rho^{(1)}(\rr)\phi(\rr)d\rr + \mu \int \rho^{(1)}(\rr)d\rr \
  = \langle K_N + V_N + k_BT \ln \mathit{f}_0 \rangle
\end{align}

It can be shown that for a given $\mu$, $T$, and defined function
$V_N$ describing the potential interaction between particles, there is
a one-to-one relation between the external potential function and the
equilibrium density profile $\rho^{(1)}$ at thermodynamic equilibrium.
The grand canonical density function $f_0$ (Eq.~\ref{eq:prob-density})
for a given function $V_N$, $\mu$, and $T$ is a function of external
potential, so that the right hand side of
Eq.~\ref{eq:intrinsic-free-two} is a function of external potential
and therefore also of $\rho^{(1)}$.  Thus intrinsic free energy $\iF$
is dependent on $\rho^{(1)}$.


We can thus write
\begin{align}
  \Omega_{\phi}[n] = \iF[n] + \int n(\rr)\phi(\rr)d\rr - \mu\int n(\rr)d\rr
\end{align}
At equilibrium, this grand potential is minimized.


The first step in Classical Density Functional Theory is to design the
functional form of the intrinisic free energy $iF$, which is
conceptually equal to the free energy ignoring the external potential.
We then set the tempurature $T$, chemical potential $\mu$, and
external potential $\phi(\rr)$, and adjust the density profile until
we have found the global minimum of this grand potential.  The result
is the density and grand potential in equilibrium.


(the statistical average of $V_N$
depends on $\rho^{(1)}$ since $V_N$ initially defined by $V_N(r^N)$ in
Eq.~\ref{eq:hamiltonian}, so a function of positions of particles, so
statistical average version is above).

Is the configuration integral, it's the potential's contribution to the partitionn function

This is the functional version of grand potential, dependent upon
$n(\rr)$, not neccesarily the equilibrium $n(\rr)$, and this will be
minimized in the case of thermodynamic equilibrium.
\begin{align}
  \Omega_{\phi}[n] = \iF[n] + \int n(\rr)\phi(\rr)d\rr - \mu\int n(\rr)d\rr
\end{align}


n-particle density in the grand canonical ensemble is:
\begin{align}
  \rho_N^{(n)}(\rr^n)=\frac{1}{\Xi}\sum^{\infty}_{N=n}\frac{exp(N\beta \mu)}{\Lambda^3 (N-n)!} \
  \int exp(-\beta V_N)d\rr^{(N-n)}
\end{align}
where the spatial integral of the Boltzmann factor is taken over all
the potential interaction with all the other particles $N$ in the
system, and all the possible $N$s are considered. The $exp(N\beta
\mu)$ boltzmann term accounts for the chemical potential's regulation
of the average number of particles in the system.  It is defined so
that an integral of the single particle density gives the average
number of particles:
\begin{align}
  \rho_N^{(1)}(\rr)d\rr = \langle N \rangle
\end{align}



canonical ensemble:
\begin{align}
  F = U - TS
\end{align}


grand canonical ensemble:
\begin{align}
  \Omega = F - N \mu
\end{align}

The second term on the right hand side normally vanishes, since the
external potential is usually a wall, so either infinity or zero, and
if it's infinity, than $n(\rr)$ there is zero.




%% \begin{figure}\caption{Smart figure}
%% \end{figure}

%% \begin{table}\caption{Smart table}
%% \end{table}

%% \section{a section}
%% I am referencing a figure~\ref{fig:fig-ref}

%% \section{a different section}
I am citing a source~\cite{huang2003dynamic}
\clearpage
\newpage

Important to note that the functional your minimizing is not the grand potential until its minimized



\section{Introduction to SAFT and explanation of first free energy term}


Work on classical Density Functional Theory for fluids involves
creating intrinsic density functionals of the density profile.  These
are comprised of a sum of terms that individually address different
conceptual aspects of the system.   terms that treat different types
of interaction between particles are added to terms that describe
reference system.  For example, the functional for the version of SAFT
that we use in our work, is defined as

\begin{align}
  iF_{SAFT} = F_{ideal} + F_{hard sphere} + F_{something}
\end{align}

We will discuss in much more detail the meaning of these terms below,
but write it down now to illustrate the general structure of the total
functional.


While our contributions described in the next chapters could certainly
be applied to many different types of fluid functionals, we
specifically worked towards improving the very popular SAFT versions
functionals in current development.  Indeed in one of the papers
described, we use the function we've created within a SAFT functional
to test it's affect upon the results of the that functional.
Therefore while describing the general background and origin of the
ideas we use, I'll speak directly to the build up of the SAFT
functional, the pieces of which (developed and verified by others) and
the general structure are found in many other functionals.


The first two terms, $F_{ideal}$ and $F_{hard sphere}$
define the reference system.  The rest of the terms address different
approximations of interactions (which we describe below).  In in
homogenous fluids, each one of these terms is itself a functional of
the density profile.

Many times different approaches will use the same reference systems
and even the exact functional form of the reference system free energy
terms while, when treating other forms of interaction, will use
different conceptual models and implementation in order to construct
those terms.  SAFT itself departs from other theories in it's last
three terms.  The first 'ideal' term it uses is ubiquitous, and the
second is a particularly common reference system for fluids.  I give a
detailed introduction to the particular functional that we use for it
below (that of the White Bear free energy), because while we don't
actually modify it in our implementation of inhomogenous SAFT, we do
draw heavily from its ideas when creating the function that we do.

The first term in the SAFT functional is the ideal free energy term
$F_{ideal}$.  This term treats a system of particles that do not
interact with eachother.  This is an obvious place to start if one is
to build the description of particle interaction terms upon a
reference system.  It's lack of interaction actually causes this term
to be the only one that we can construct exactly, with no
approximations.  To see why, we point out that a system of non
interacting spheres is able to satisfy the local density apporximation
(only here it is not an approximation!)  The local density
approximation states that a free energy functional can be written as
an integral of a completely local function of the density profile.
For example,
\begin{align}
  \iF_{local~density}[\rho^{(1)}] = \int f(\rho^{(1)}) d\rr
\end{align}

where $f(\rho^{(1)})$ is the free energy per unit volume of the
homogenoues fluid at a density $\rh$.  This approximation neglects any
interaction between the particles, so that any spatial variation in
the density will be due entirely to the varying external potential.
In essence, each bit of volume becomes it's own thermodynamic system,
with a free energy equal that of a homogenous density of particles,
and the free energies from all the bits of volume are added up to get
the total.  This approximation has also been used to construct entire
intrinsic free energy functional (cite). In fact it does apply to
external potentials that modulate the density slowly over space, much
more slowly than correlation lengths.  The approximation breaks down
rather quickly, however, near hard walls for interacting systems,
where often the spheres will stack up in 'layers' (should see some
figure) and the densities are higher than bulk freezing densities
(cite).

However, the ideal term can be constructed in this fashion, so all we
need to do is integrate the free energy of an ideal homogenous system
at the local density.  Going back to basic thermodynamics and
statistical mechanics, we have:
\begin{align}
  F = -k_BT \ln Q_N = -k_BT \ln (\frac{V}{N!\Lambda^3})
\end{align}
where $Q_N$ is the partition function and $\Lambda$ is de Broglie
thermal wavelength.  Using th Stirling approximation for $N!$, we have
\begin{align}
F^{id} = Nk_BT(\ln \Lambda^3 \rho - 1))
\end{align}

Taking this as the free energy per unit volume and integrating, we
have
\begin{align}
  \beta \iF[\rho] = \int d\rr \rho(\rr)(\ln (\Lambda^3\rho(\rr))-1)
\end{align}

Because this term is in fact an exact one for the ideal gas, it is
ubiqitous in the use of classical density functional theory for
inhomogeneous and homogeneous fluids.

\clearpage
\newpage

\section{Virial Equation, Mayer functions, and the Carnahan Starling Equation}

The second term in the SAFT intrinsic free energy functional,
$f_{hard~sphere}$, is also a reference term and is used in many other
functionals besides the one described here.  I'll introduce the term
and the theory behind it in some detail in the next section because it
is so widely used and, more importantly, because an understanding of
the ideas involved is neccesary for an understanding of our own work.
Before describing the term itself, however, I'll explain here some
important things that lead up to this theory, namely the Virial
equation, Mayer functions, and the Carnahan Starling Equation.

The Virial Equation applies to homogeneous fluids.  It equates a
thermodynamic, intensive quantity (likes the pressure) with an
expansion of the homogeneos density of the fluid.  It's standard form
is
\begin{align}
  \frac{\beta P}{\rho} = 1 + \sum_{i=1}^{\infty}\beta_i\eta^i.
\end{align}

The expansion comes out of a formulation of the partition function
that is most often expressed as a series of diagrams that have well
defined rules of construction.  I won't explain the diagrams or their
rules here, but Figure(REF) shows an example so that if the reader
sees them somewhere she'll know what they are.  A single term (or
diagram) in this expansion of the partition function is in fact a
spatial integral of an integrand composed of particle densities
multiplied by a number of what are called Mayer Functions.  To
explain, I'll start with the partition function:
\begin{align}
  \Xi &= \sum_{N=0}^{\infty}\frac{exp(N\beta\mu)}{h^{3N}N!}\int...\int exp(\beta \mathcal{H})d\rr^Nd\mathbf{p}^N \\
  &=\sum_{N=0}^{\infty}\frac{exp(N\beta\mu)}{h^{3N}N!}\int...\int exp(-\beta (V_N + \phi(\rr))d\rr^Nd\mathbf{p}^N
\end{align}
where $V_N$ is the interaction potential between all the particles in
the system and $\phi(\rr)$ is the external potential.  If the the
interaction potential can be written as a summation of pairwise
superposable interactions, i.e.
\begin{align}
  V_N = \sum_{i<j}^{all~particles} v(i,j)
\end{align}
 than the partition function can be written as
\begin{align}
  \Xi = \sum_{N=0}^{\infty}\frac{1}{N!}\int... \int \left(\overset{N}{\underset{i<j}{\Pi}} f(i,j)\right)
  \left( \overset{N}{\underset{i=1}{\Pi}} \frac{exp(\beta (\mu - \phi(\rr)))}{\Lambda^3}\right)d\rr^N
\end{align}
where $f(i,j)=exp(-\beta v(i,j))$ is the Mayer function between two
particles.  With the help of the diagrams, one can take the natural
logarithm of this function to get the grand potential, and then
derivatives to find what are called direct correlation functions
(which I won't explain here). The use of an identity yields a
relationship between the chemical potential and the density:
\begin{align}
  \beta \mu = \beta \mu^{id} - \sum_{i=1}^{\infty}\beta_i\rho^i
\end{align}
Then the use of the following relation from thermodynamics:
\begin{align}
  \left(\frac{\partial P}{\partial \rho}\right)_T \
  = \rho \left(\frac{\partial \mu}{\partial \rho}\right)_T
\end{align}
allows one to express the pressure as
\begin{align}
  \frac{\beta P}{\rho} = 1 + \sum_{i=1}^{\infty}\beta_i\eta^i
\end{align}
where
\begin{align}
  \label{eq:virial-coeff}
  \beta_i = (\frac{6}{\pi d^3})^i B_{i+1}
\end{align}


(REWRITE BELOW)
The above formulation of a thermodynamic property is useful, but it
requires an expansion of coeffients, which can be a nuisance
computationally.  Caranahan and Starling (cite) were able to develop a
rule for coefficient generation that approximates
Eq.~\ref{eq:virial-coeff}, but that results in integers, and ones that
one can allow for a geometric the one written above, based on and then
multiple terms of a The virial coefficients $\beta_i$ based on the
procedure above are not integers, so in order to get get to a
thermodynamic property the series must be expanded.  Carnahan Starllng
constructed a method that allows them to approximate each coefficient
by a very close integer in a sysematic fashion that allows them to
create a geometric series.  This series yields a simple, analytic
approximation of the pressure in the homogenous fluid as a function of
density 3.9.17


\begin{align}
  \frac{\beta P}{\rho}=\frac{1+\eta+\eta^2-\eta^3}{(1-\eta)^3}
\end{align}
or for the free energy:
\begin{align}
  \frac{\beta F^{ex}}{N}=\frac{\eta(4-3\eta)}{(1-\eta)^2}
\end{align}

Its is very successful in predicting pressure of homogenous hard
sphere fluid at different densities.

White Bear uses MCSL equation of state, which is the generalization to
the multi component mixture of hard spheres of the Carnahan-Starling
equation of state.  We describe the Carnahan-Starling equation of
state here because we use it in the first chapter.





\section{$F_{hard~sphere}$, Fundemental Meaure Theory, and White Bear}

We now return to analyzing the terms within SAFT:
\begin{align}
  F_{SAFT} = F_{ideal} + F_{hard~sphere} + F_{other terms}
\end{align}

Remember that $F_{ideal}$, because it treats particles that do not
interact with one another, can be described as an intagral of a purely
local free energy function over the volume of the system.  In fact,
this one term addresses any aspect of the system that is
non-interactive, in the sense that every other term in $F_{SAFT}$ is
designed to specifically deal with a different type of potential
interaction among the particles.  Thus, $F_{ideal}$ is in a sense the
most basic reference term in the system.  However, as I've explained
above, the seconds term, $F_{hard~sphere}$, which we will describe here,
is also considered to be a reference system.

The potential interaction it describes is based on the fact that every
particle has at it's core a 'hard-core' repulsion to any other
particle.  In other words, as two particles approach each other in
space, there is a sudden, sharp spike in potential that prevents the
two particles from 'overlapping'.  The forces here can be complicated,
deriving from nuclear forces and the exclusion principle, but our
classical theories seek to approximate these in simple ways.  The
$F_{hard~sphere}$ potential formulation that we use is not the only
way to address this.  The Leonard-Jones potential, for example,
another widely used potential energy description, approximates the
repulsion with a positive term porportional to $\frac{1}{r^{12}}$.

The hard sphere potential interaction is characterised by an
impenetrable spherical volume that is centered at the particles'
position.  The potential between two of these hard spheres is:
\begin{align}
  v(r) &= \infty,~~ r < r_A + r_B \\
  &= 0,~~ r > r_A + r_B
\end{align}
where $r$ is the distance between the two particles, and $r_A$ and
$r_B$ are the radii of the two particles.

Some of the intial attempts to get around this included expanding this
local free energy in powers of dereivitives of the density with
respect to positions (but maybe not mention this?).  Also some of the
initial attempts included what was called weighted density
approximation

A weighted density $\bar{\rho}$ is created through a convolution of
the $\rh$:
\begin{align}
  \bar{\rho}(\rr) = \int w(|\rr-\rr'|) \rh(\rr') d\rr'
\end{align}
and then the intrinsic free energy is defined as
\begin{align}
  \iF[\rh] = \int \frac{f^{ex}(\bar{\rho})}{\bar{\rho}} \rh(\rr') d\rr'
\end{align}


This hard sphere reference is described by an interaction
potential that discontinuously jumps from zero to infinity when the
spheres are within a certain range (the diameter of the hard
sphere).  This reference system potential is meant to approximate
the 'hard core' repulsions that are common to all particles.  It
seems to dos some stuff well (WHAT DOES IT DO WELL) Together with
the ideal free energy term (described below),



Fundemental measure Theory, creatd by Rosenfeld in 1989, is based on
convolutions, but represents a considerable departure.

Percis-Yevik have an involved derivation of an equation of state,
which once again allows one to write thermodynamic properties in terms
of densities for homogeneous systems.  Comparing this theory (and
using properties of correlation functions, which we will once again
not get into) with the derivations of another theory call Scaled
Particle Theory, which once again we will not explain but will say
that it has to do with a system of spheres where one sphere is slowly
growing, Rosenfeld recognized that the density sides of these
equations can be reformulated in terms of combinations of convolutions
of weighting functions that are constructed to describe the geometric
properties of spheres.  He then discovered that for inhomogeneous
systems, he could write down the intrinsic free energy in terms of
convolutions of densities with these weighting functions in such a way
that the Percis-Yevik correlation functions were reproduced.  In his
derivation, the intrinsic free energy neccesarily has the form:
\begin{align}
  \label{eq:F-FMT-form}
  \iF^{ex}[\rho^{(1)}] = \int \Phi^{ex}({n_i(\rr')})d\rr'
\end{align}
It's a function of fundemamental measures evaluated at the point
$\rr$.  Dimensional analysis throughout the derivation places further
restrictions on the form of Eq.~\ref{eq:F-FMT-form}, but at the end of
the day there is still some freedom in how the functional is designed.
As I said before, the Perkis-Yevik derivation yields an equation of
state, which can related the Free energy to the density in the
homogeneous limit.  In the original formulation, Rosenfeld chose to
design his functional so that it reproduces this equation in the limit
of a homogeneous fluid.

 It originates
from the fact that the mayer function that goes into describing the
virial expansions can itself be deconvolved into a sum of products of
convolutions of weighting functions that are designed with geometry of
the hard sphere in mind.(should probably describe the virial equation
before this, actually).  The resulting expression for the excess free
energy (excess to the ideal free energy) can be formaulted in terms of
a series of closely related fundemental measures, that also reflect
the geometry of the hard sphere reference system.  They are the
following:

The White Bear free
energy is thus
\begin{equation}
A_\textit{HS}[n] = k_B T \int \left(\Phi_1(\rr) + \Phi_2(\rr) + \Phi_3(\rr)\right) d\rr \; ,
\end{equation}
with integrands
\begin{align}
\Phi_1 &= -n_0 \ln\left( 1 - n_3\right) \label{eq:Phi1}\\
\Phi_2 &= \frac{n_1 n_2 - \mathbf{n}_{V1} \cdot\mathbf{n}_{V2}}{1-n_3} \\
\Phi_3 &= (n_2^3 - 3 n_2 \mathbf{n}_{V2} \cdot \mathbf{n}_{V2}) \frac{
  n_3 + (1-n_3)^2 \ln(1-n_3)
}{
  36\pi n_3^2\left( 1 - n_3 \right)^2
} , \label{eq:Phi3}
\end{align}
using the fundamental measures
\begin{align}
  n_3(\rr) &= \int n(\rr') \Theta(\sigma/2 -\left|\rr - \rr'\right|)
  d\rr' \label{eq:FMn3} \\
  n_2(\rr) &= \int n(\rr') \delta(\sigma/2 -\left|\rr - \rr'\right|) d\rr' \\
  \mathbf{n}_{2V}(\rr) &= \int n(\rr') \delta(\sigma/2 -\left|\rr - \rr'\right|) \frac{\rr-\rr'}{|\rr-\rr'|}d\rr'
\end{align}
\begin{align}
  \mathbf{n}_{V1} = \frac{\mathbf{n}_{V2}}{2\pi \sigma}, \quad
  n_1 &= \frac{n_2}{2\pi \sigma} , \quad
  n_0 = \frac{n_2}{\pi \sigma^2} \label{eq:FMrest}
\end{align}

where $\delta$ is the direc delta function and $\Theta$ is the
Heavyside step function.



Fundemental Measure theory comes out of a similar result of the
Percus-Yevick equation of state and the result of scaled particle
theory

There are very accurate equatio

I think do introduction to what the fundemental theory does, here,
with the weighted densities based on geometry.

Based on the geometry of particles in question.  Can deal with
mixtures...  Below we will describe in more detail the White Bear
functional, which is what we use for our term that approximated hard
spheres.

The derivation of Fundmental Measure Theory is inspired by a
similarity between the results of a homogenous theory called scaled
particle theory and an equation of state called the Perkis Yevik
equation.  I won't explain the derivation, but we will mention some
things about the Perkis Yevik equation

When constructing a functional theory that will be appropriate for
inhomogenous fluids, Rosenfeld had the freedom to dictate what the
functional would say in the limit of a homogenous fluid.  A great deal
of work has been done with homogenous fluids and fluids mixtures.
This work results in equations of state, which relate thermodynamic
properties of homogenous fluids such as free energy pressure or free
energy to the density.  Rosenfeld chose to build up his intrinsic free
energy functional so that in the limit of homogenous distribution it
approached that given by the Percus-Yevik equation of state.  This is
a successful equation of state.

The result was the very successful Fundemental Measure Theory (in what
ways was it successful?)

Later Roth et al. created there own fundemental measure functional
that is in the limit of homogeneous density approaches not the Percis
Yevik equation but instead the Carnahan-Starling equation of state.
They do so because for the hard sphere fluid at high densities,
approaching freezing, the pressure is over estimated in Percis Yevik.
FMT follows the contact theorem, that $\rho_w= \beta p$ where $\rho_w$
is the contact density of spheres in contact with a hard wall, which
means that this value is consequently overestimated.  White Bear
people keep many of the ideas in place, but replace the Percus-Yevik
underlying equation of state to that of
Mansoori-Carnahan-Starling-Leland: (reference below) As a consequence,
the White Bear functional is extremely successful in predicting the
density profiles of inhomogenous collections of hard spheres
(examples?).

The diagramatic expression approach gives the virial equation and it's
exact in low-density limit


The percis-Yevik equation of state that the original FMT is based on
is not the low density limit, say that the excess free energy function
istelf is built so that the thermodynamic equation of state of hard
spheres is that given by the Perkis-Yevik equation of state.

This is the FMT that we use in out functionals so it is the one that
we will briefly describe.  Fill the first part of this out a bit more,
ore just explain more how you'd be able to get these things.

\section{The convolution theorem}

One of the largest advantages to using Fundamental Measure Theory, and
one that is not immediately obvious, is that the convolutions that
combine to construct the functional allow for very efficient
computation.  This is not very intuitive, since an integral of a
convolution integral must integrate over two dimensions, e.g.
\begin{align}
\int(f\ast g)(\xx)d\xx = \int \int f(\yy)g(\xx-\yy)d\yy d\xx
\end{align}
so that it may seem that the size of the computation would scale as
$N^2$, where $N$ is the size of the system.  It is true that in the
case of FMT, the weighting functions cut off the integrals at the size
on the order of a sphere of particle radius, but this can still be a
large enough volume so that a double integral for which one of the
volumes is this size and the other is the size of the whole system
would be too costly for practical computation.  FMT is saved, however,
by what is called the Convolution Theorem.  The Convolution Theorem
states that when one takes the Fourier Transform on a spatial
convolution of two functions (a double integral in space), the result
is two separate integrals over k-space that are simply multiplied
together.  (SHOW IN EQ FORM) When minimizing our functional, after
taking a Fourier Transform of the convolutions, we have merely to
integrate once over the one variable $\kk$.  Wait!  You may say.  This
is a cheat, since the Fourier Transform is itself an integral, so at
the end of the day we're still performing two integrals.  This would
be true if it were not for the computational technique known as Fast
Fourier Transforms, developed by (cite).  I won't explain how it works
here, but it's effect is to perform the transform, which would
normally have a computational cost on the order of N, with a
computational cost on the order of $\ln N$ instead.  Thus, when we
Fourier Transform the convolutions in the functional and proceed to
take the single integral over the result (in k-space), the
computational cost scales as $N \ln N$.  For large systems (we
simulate systems with hundreds or even thousands of particles) this
can certainly be the difference between practically possible and
impossible computations.  The proof is short and pretty so I'll relate
it here:
\begin{align}
\hat{h}(\kk) = \hat{f}(\kk)\hat{g}(\kk)
\end{align}
Is the convolution theorem.  Proof:
\begin{align}
h(\xx) &= (f\ast g)(x) = \int f(\yy)g(\xx-\yy)d\yy \\
\hat{f}(\kk) &= \int f(\yy) exp(-i2\pi \kk \cdot \yy)d\yy \\
\hat{g}(\kk) &= \int g(\yy) exp(-i2\pi \kk \cdot \yy)d\yy \\
\int h(\rr) exp(-i2\pi \kk \cdot \zz) d\zz &= \int \int f(\yy) g(\zz-\yy) exp(-i2\pi \kk \cdot \zz) d\yy d\zz
\end{align}
because the two integrals are necessarily over all space,
\begin{align}
\xx &= \zz - \yy \\
\hat{h}(\kk) &= \int \int f(\yy) exp(-i2\pi \kk \cdot \yy ) d\yy g(\xx) exp(-i2\pi \kk \cdot \xx) d\xx \\
&= \hat{f}(\kk) \hat{g}(\kk)
\end{align}


\section{Contact Value Theorem}

Fixed number of particles:
\begin{align}
F &= U - TS \\
dU &= TdS -pdV \\
dF &= dU - TdS - SdT = -SdT - pdV
\end{align}
picture a fixed volume except a little protrusion into it, so call
$F_{out}$ and $F_{flat}$.  So looking at partition functions:
\begin{align}
Z_{flat} = \int_V... \int_V exp(-\phi \{\rr^N \})d\rr^N
Z_{out} =  \int_{V-dV}... \int_{V-dV} exp(-\phi \{\rr^N \})d\rr^N
\end{align}
Writing $Z_{flat}$ in terms of $Z_{out}$:
\begin{align}
Z_{flat} = &\int_{V-dV}... \int_{V-dV} exp(-\phi \{\rr^N \})d\rr^N \notag \\
+ &\int_V... \int_V \int_{V-dV} exp(-\phi \{ \rr^N \})d\rr^N \
+ \int_V ... \int_{V-dV} \int_V exp(-\phi \{ \rr^N \})d\rr^N ... \notag \\
+ &\int_V... \int_{V}\int_{V-dV} \int_{V-dV} exp(-\phi \{\rr^N \})d\rr^N \
+ \int_V... \int_{V-dV}\int_V \int_{V-dV} exp(-\phi \{\rr^N \})d\rr^N ...
\end{align}
The term on the first line is just $Z_{out}$.  The terms on the second
line treat one particle being in the bit of volume and the others
being integrated over the rest of the volume.  Becuase every particle
is identical, it doesn't matter which is in the little bit of volume,
each of these terms is the same.  So they can be replaced by one of
them multiplied by $N$.  The terms on the last line or any line after
are of order 2 or higher in $dV$.  We will throw away these terms, and
there are two arguments that allow us to do this.  The simplest one is
that $dV$ is small, and so terms with a two $dV$s multiplied together
will be small.  One must be careful with this, however, because one is
really intagrating over the boltzmann factor over these volumes.  If
the interaction potential between the particles is constructed so that
they are highly attracted to eachother, than a state in which there
are two particles in the bit of volume can have the same order of
magnitude probability as the one in which there is just one.  In this
case we would not be able to say for certain that these terms are so
much smaller than the order 1 $dV$ terms on the seconds line that we
could reasonably ignore them. Thus the validity of the derivation has
to do with the size of your smallest measurement along the wall, and
the nature of the attractive potentialbetween the particles.  In our
case we deal with an interaction between hard spheres, which simply
exclude other spheres from being too close to them.  Thus it's
reasonable for us to imagine that if there is one hard sphere in the
bit of volume, than there is only the one, and any terms that address
the situation in which there are two in the bit of volume can be
ignored.  After applying these areguments we have:
\begin{align}
  \label{eq:zflat}
  Z_{flat} = Z_{out} + N \int_V... \int_V \int_{V-dV} exp(-\phi \{ \rr^N \})d\rr^N
\end{align}

Now we consider the statistical mechanical definition of the particle
density,
\begin{align}
  n(\rr) = \frac{N \int_V... \int_V exp(-\phi \{ \rr^N \})d\rr^{N-1}}{\int_V... \int_V exp(-\phi \{ \rr^N \})d\rr^N}
\end{align}
Which is the sum of the boltzmann factors of all the states for which
a particle is at position $\rr$, divided by the partition function for
the system.  Comparing this with the right-most term in the
Eq.~\ref{eq:zflat} we see that:
\begin{align}
 N \int_V... \int_V \int_{V-dV} exp(-\phi \{ \rr^N \})d\rr^N  &= Z_{flat} \int_{dV} n(\rr) d\rr
 &\approx Z_{flat} n(\rr)dV
\end{align}
where on the right we make the approximation that because the volume
is small $n(\rr)$ is constant over it.
Thus we have
\begin{align}
Z_{flat} = Z_{out} + Z_{flat} n(\rr)dV \\
Z_{out} = Z_{flat} (1-n(\rr)dV)
\end{align}

We have
\begin{align}
  dF = F_{out} - F_{flat} &= -kT \ln \left( \frac{Z_{out}}{Z_{flat}} \right) \notag \\
  &= -kT \ln \left( \frac{Z_{flat} (1-n(\rr)dV)}{Z_{flat}} \right) \notag \\
  &= -kT \ln \left( 1-n(\rr)dV \right) \notag \\
  &= kTn(\rr)dV
\end{align}
Then relating back to thermodynamics:
\begin{align}
  pdV &= dF = ktn(\rr)dV \\
  \label{eq:contact-value-theorem}
  p &= ktn(\rr)
\end{align}
Eq.~\ref{eq:contact-value-theorem} is the standard formulation of the
contact value theorem.


Another equation for the particle density is:
\begin{align}
n(\rr) = \frac{\delta \iF}{\delta V_{ext}}
\end{align}

\clearpage
\newpage




\section{Remaining terms in SAFT}
